\documentclass{uebblatt}
\usepackage{enumitem}

\begin{document}

\maketitle{9}

\begin{aufgabe}{Ableiten und Integrieren als Endos im $\mathbb{R}$-VR $\mathbb{R}[X]$ (2+2+2+2=8 Punkte)}
Es sei $\mathbb{R}[X]$ der Polynomring über dem Körper $\mathbb{R}$ (mit einer Unbestimmten); $\mathbb{R}[X]$ ist bekanntlich unendlich dimensionaler $\mathbb{R}$-Vektorraum. Wir bezeichnen diesen Vektorraum kurz mit $V$. Folgende Abbildungen $D$ und $I$ von $V$ nach $V$ werden betrachtet:
$$D: V \to V, \> p(X) = \sum \limits_{k=0}^{n} a_k X^k \mapsto \sum \limits_{k=1}^{n} ka_k X^{k-1} = p'(X)$$
$$I: V \to V, \> p(X) = \sum \limits_{k=0}^{n} a_k X^k \mapsto \sum \limits_{k=0}^{n} \frac{1}{k+1} a_k X^{k+1} = \int \limits_{0}^{X}p(t) \, dt$$
\begin{enumerate}
\item $D$, $I \in Hom(V, V)$.
\item $D$ ist surjektiv, aber nicht injektiv.
\item $I$ ist injektiv, aber nicht surjektiv.
\item Man bestimme explizit die Abbildungsvorschrift der Hintereinanderausführungen $D \circ I$ und $I \circ D$.
\end{enumerate}
\underline{Ergänzende Bemerkung}:\\
Für einen endlich dimensionalen $K$-Vektorraum $W$ und eine $K$-lineare Abbildung $f : W \to W$ gilt „$f$ surjektiv $\Leftrightarrow$ $f$ injektiv $\Leftrightarrow$ $f$ bijektiv“. b) und c) zeigen, dass  „…“ nicht auf unendlich dimensionales $W$ ausgeweitet werden kann.
\end{aufgabe}


\begin{aufgabe}{Die Elemente der Diedergruppe als Endos im $\mathbb{R}$-Vektorraum $\mathbb{R}^2$ (4+2=6 Punkte)}
An der Beschreibung der Dieder-Gruppe $D_4$ in \textbf{Aufgabe 9} erkennt man, dass jedes Element von $D_4$ eine $\mathbb{R}$-lineare Abbildung von $\mathbb{R}^2$ nach $\mathbb{R}^2$ ist; jedes Element von $D_4$ ist von der Form\\
$f : \mathbb{R}^2 \ni x \mapsto Ax \in \mathbb{R}^2$, wobei $A = M(f; B, B) \in \mathbb{R}^{2 \times 2}$ mit der Standardbasis $B = (e_1, e_2)$ von $\mathbb{R}^2$.
\begin{enumerate}
\item Für $R_i, S_k$ ($i = 0, 1, 2, 3$, $k = 1, 2, 3, 4$) bestimme man die Matrizen $M(R_i; B, B), M(S_k; B, B)$.
\end{enumerate}
\underline{Hinweis vorab zu b)} (Er soll nicht nachgewiesen werden, dient vielmehr als Grundlage für b)!): \\
Für $n \in \{2, 3, 4, …\}$ bilden die $n \! \times \! n$-Matrizen mit Komponenten in $\mathbb{R}$ (also insbesondere die $2 \! \times \! 2$-Matrizen mit Komponenten in $\mathbb{R}$) mit der Multiplikation $A \cdot A' = (a_{ij}) \cdot (a'_{ij}) = (c_{ij})$, $c_{ij} \defeq a_{i1} a'_{1j} + a_{i2} a'_{2j} + … + a_{in} a'_{nj}$ ein Monoid mit dem neutralen Element $E_n = (\delta_{ij}), \delta_{ij} \defeq 0$ für $i \neq j, \delta_{ii} \defeq 1$. 
\begin{enumerate}[resume]
\item Man zeige, dass die acht Matrizen von a) bezüglich der Multiplikation von Matrizen eine nichtabelsche Gruppe bilden.
\end{enumerate}
Hilfe: Es empfiehlt sich die Herstellung eines Bezugs zu dem, was bereits in \textbf{Aufgabe 9} gezeigt wurde.
\end{aufgabe}

\newpage

\begin{aufgabe}{Ein Beispiel für einen Endomorphismus von $\mathbb{R}^3$ aus dem Bereich der Wirtschaft (2+1+2=5 Punkte)}
Drei Firmen A, B, C mit den Marktanteilen $\alpha_0, \beta_0, \gamma_0[\%]$ ($\alpha_0 + \beta_0 + \gamma_0 = 100$) konkurrieren. Nach einer Werbekampagne werden nach einem Jahr folgende Kundenbewegungen festgestellt:\\
A hat $5\%$ seiner Kunden an B und $10\%$ seiner Kunden an C verloren, B hat $15\%$ an A und $10\%$ an C verloren, C hat jeweils $5\%$ an A und B verloren.\\
(Anders ausgedrückt: A hat beispielsweise nur $85\%$ seiner Vorjahreskunden behalten, aber $15\%$ der Vorjahreskunden von B und $5\%$ der Vorjahreskunden von C dazu gewonnen.)
\begin{enumerate}
\item Welche Marktanteile $\alpha_1, \beta_1, \gamma_1 [\%]$ ($\alpha_1 + \beta_1 + \gamma_1 = 100$) haben die Firmen A, B, C nach einem Jahr? \\
Man formuliere dies als lineares Gesetz, d.h. als eine $\mathbb{R}$-lineare Abbildung\break $\mathbb{R}^3 \ni x \mapsto Mx \in \mathbb{R}^3$ mit einer Matrix $M \in \mathbb{R}^{3 \times 3}$, so dass $\begin{pmatrix}\alpha_1\\\beta_1\\\gamma_1\end{pmatrix} = M \begin{pmatrix}\alpha_0\\\beta_0\\\gamma_0\end{pmatrix}$
\end{enumerate}
Der gleiche Trend setzt sich auch in den kommenden Jahren fort.
\begin{enumerate}[resume]
\item Man berechne, welche Marktanteile die Firmen A, B, C nach zwei und drei Jahren haben.
\item Man ermittle, für welche Marktanteile ($\alpha_0, \beta_0, \gamma_0$) dieser Trend nichts ändern würde.
\end{enumerate}
\end{aufgabe}


\end{document}