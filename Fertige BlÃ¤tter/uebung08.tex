\documentclass{uebblatt}

\begin{document}

\maketitle{8}

\begin{aufgabe}{Anwendung des Austauschlemmas}
Man mache im $\mathbb{R}$-Vektorraum $\mathbb{R}^4$ durch zweimalige Anwendung des Austauschlemmas aus der Standardbasis $(e_1, e_2, e_3, e_4)$ eine Basis von $\mathbb{R}^4$, deren erster Basisvektor $v_1 = (2, -1, 3, -2)$ und deren zweiter Basisvektor $v_2 = (3, 2, -6, 1)$ ist.
\end{aufgabe}


\begin{aufgabe}{Zur Austauschbarkeit von Vektoren}
Kann man im $\mathbb{R}$-Vektorraum $\mathbb{R}^4$ zwei der Vektoren $(1, 1, 0, 0), (1, 0, 0, 1), (0, 1, 1, 0), (0, 0, 1, 1)$ durch $(2, -3, -2, 3), (1, -1, 1, 3)$ ersetzen, so dass die sich dadurch ergebenden vier Vektoren linear unabhängig sind? Man begründe die Antwort.
\end{aufgabe}


\begin{aufgabe}{Eine interessante $\mathbb{R}$-lineare Abbildung in $\mathbb{R}^2$}
Es sei im $\mathbb{R}$-Vektorraum $\mathbb{R}^2$ die Basis $(a_1, a_2)$ mit $a_1 = (1, 0)$ und $a_2 = (1, 1)$ gegeben. Die $\mathbb{R}$-lineare Abbildung $A: \mathbb{R}^2 \to \mathbb{R}^2$ wird durch lineare Fortsetzung von
\begin{center}
$Aa_1 = a_2$ und $Aa_2 = a_1$ ($Aa_1$ bzw. $Aa_2$ Kurzschreibweise für $A(a_1)$ bzw. $A(a_2)$)
\end{center}
definiert.
\begin{enumerate}
\item Man bestimme die Untervektorräume $U_1 , U_2$ von $\mathbb{R}^2$, für die gilt:\\
$Ax = x$ für alle $x \in U_1$, $Ax = -x$ für alle $x \in U_2$.\\
Man skizziere diese in der Ebene $\mathbb{R}^2$.
\item Man gebe an, wie man für $x \in \mathbb{R}^2$ das Bild $Ax \in \mathbb{R}^2$ konstruiert. Man skizziere geeignete Parallelogramme.
\item Offensichtlich gilt $A^2 = id$. Ist $A$ geometrisch eine Spiegelung? ($A^2$ steht kurz für $A \circ A$ und $id$ ist die identische Abbildung $\mathbb{R}^2 \to \mathbb{R}^2$, $x \mapsto x$)
\end{enumerate}
\end{aufgabe}


\begin{aufgabe}{Projektionen als wichtige $\mathbb{R}$-lineare Abbildungen in $\mathbb{R}^2$}
Es sei im $\mathbb{R}$-Vektorraum $\mathbb{R}^2$ die Basis $(a_1, a_2)$ wie in \textbf{Aufgabe 26} gegeben. Die $\mathbb{R}$-lineare Abbildung $A: \mathbb{R}^2 \to \mathbb{R}^2$ wird durch lineare Fortsetzung von
\begin{center}
$Aa_1 = a_1$ und $Aa_2 = 0$
\end{center}
definiert.
\begin{enumerate}
\item Man bestimme $im(A)$ und $ker(A)$ und skizziere die Untervektorräume in $\mathbb{R}^2$.
\item Man gebe an, wie man für $x \in \mathbb{R}^2$ das Bild $Ax \in \mathbb{R}^2$ konstruiert. Man bestätige $A^2 = A$ und interpretiere die Aussage „$A$ projiziert $x$ auf $im(A)$ längs $ker(A)$“.
\end{enumerate}
\end{aufgabe}


\end{document}