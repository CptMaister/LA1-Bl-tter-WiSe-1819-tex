\documentclass{uebblatt}

\begin{document}

\maketitle{6}

Hinweis: $\mathbb{N}$ ist die Menge der natürlichen Zahlen, $\mathbb{Q}$ die der rationalen, $\mathbb{R}$ die der reellen.

\begin{aufgabe}{Zum $\mathbb{R}$-Vektorraum $\mathbb{R}^3$}
Im $\mathbb{R}^3$, der $\mathbb{R}$-Vektorraum ist, sind folgende Mengen gegeben:\\
$S_1 = \{x = (\alpha_1, \alpha_2, \alpha_3) | \alpha_1 = \alpha_2 = \alpha_3\}$\\
$S_2 = \{x = (\alpha_1, \alpha_2, \alpha_3) \> | \> \alpha_3 = 0\}$\\
$S_3 = \{x = (\alpha_1, \alpha_2, \alpha_3) \> | \> \alpha_1 = \alpha_2 - \alpha_3\}$\\
$S_4 = \{x = (\alpha_1, \alpha_2, \alpha_3) \> | \> \alpha_1 = 1\}$
\begin{enumerate}
\item Man skizziere die Mengen $S_i$ innerhalb von $\mathbb{R}^3$, $i = 1, ..., 4$.
\item Man bestimme die Untervektorräume $U­_i = span(S_i)$, $i = 1, ..., 4$.
\item Man gebe für alle Untervektorräume $U_i$, $i = 1, ..., 4$, Basen an.
\item Man bestimme die Untervektorräume $W_{ij} = U_i + U_j$, $i, j = 1, ..., 4$.\\
Welche der Summen $U_i + U_j$ sind direkt, d.h. $U_i + U_j = U_i \oplus U_j$?
\end{enumerate}
\underline{Ergänzende Bemerkung}:\\
Im $\mathbb{R}^3$ - jede Basis von $\mathbb{R}^3$ besteht aus drei Vektoren - sollte man immer folgende anschauliche Vorstellung haben: $(x­_1,x_2,x_3)$ ist genau dann eine Basis von $\mathbb{R}^3$, wenn $x_1$ nicht $(0,0,0)$ ist, $x_2$ nicht auf der Geraden durch $(0,0,0)$ und $x_1$ liegt und $x_3$ nicht in der Ebene durch\break$(0,0,0)$, $x_1$ und $x_2$ liegt. 
\end{aufgabe}


\begin{aufgabe}{Zum $\mathbb{R}$-Vektorraum $\mathbb{R}^4$}
Welche der folgenden Familien von Vektoren im $\mathbb{R}$-Vektorraum $\mathbb{R}^4$ sind linear unabhängig, ein Erzeugendensystem oder eine Basis? Man begründe die Antworten.
\begin{enumerate}
\item $((1, 1, 1, 1), (1, 0, 0, 0), (0, 1, 0, 0), (0, 0, 1, 0), (0, 0, 0, 1))$
\item $((1, 0, 0, 0), (2, 0, 0, 0))$
\item $((17, 39, 25, 10), (13, 12, 99, 4), (16, 1, 0, 0))$
\item $((1, \frac{1}{2}, 0, 0), (0, 0, 1, 1), (0, \frac{1}{2}, \frac{1}{2},1), (\frac{1}{4}, 0, 0, \frac{1}{4}))$
\end{enumerate}
\end{aufgabe}


\begin{aufgabe}{Zum $\mathbb{Q}$-Vektorraum $\mathbb{R}$}
\begin{enumerate}
\item Man bestätige, dass $\mathbb{R}$ ein $\mathbb{Q}$-Vektorraum ist.
\item Ist die Familie $(1, \sqrt{2}, \sqrt{3})$ im $\mathbb{Q}$-Vektorraum $\mathbb{R}$ linear abhängig oder linear unabhängig? Man begründe die Antwort.
\end{enumerate}
\end{aufgabe}


\begin{aufgabe}{Zum $\mathbb{R}$-Vektorraum $\mathbb{R}^\mathbb{N}$}
Es ist $\mathbb{R}^\mathbb{N}$ – auch die Bezeichnung $\mathbb{R}^\infty$ ist gebräuchlich – der $\mathbb{R}$-Vektorraum wie in\break\textbf{Aufgabe 16}; $e_k \in \mathbb{R}^\mathbb{N}$ sei wie folgt definiert: $e_k =  (0, … ,1,0,…)$ mit der $1$ an der\break k-ten Stelle. Es sei $B \defeq (e_1 ,e_2 ,…)$.
\begin{enumerate}
\item Man zeige, dass die Familie $B$ im $\mathbb{R}$-Vektorraum $\mathbb{R}^\mathbb{N}$ linear unabhängig ist.
\item Was ist $span(B)$? Man begründe die Antwort.
\end{enumerate}
\end{aufgabe}



\end{document}