\documentclass{uebblatt}

\begin{document}

\maketitle{10}

\begin{aufgabe}{?????}
Es sei $V$ der $\mathbb{R}$-Vektorraum $\mathbb{R}^2$ und $B=(e_1, e_2)$ die Standardbasis von $V$. $f \in Hom(V, V)$ sei die Drehung um $0_V$ im Gegenuhrzeigersinn um den Winkel $\varphi$. Man bestimme die Matrix $M(f; B, B) \in \mathbb{R}^{2 \times 2}$.\\
Hilfe: Man denke an die geometrischen Bedeutungen von Sinus und Kosinus.
\end{aufgabe}


\begin{aufgabe}{?????}
\begin{enumerate}
\item Man skizziere im $\mathbb{R}^2$ mit der Standardbasis $B = (e_1, e_2)$ die Vektoren 
$$\tilde{a_1} = \frac{1}{2} \, e_1 + \frac{1}{2} \sqrt{3} \, e_2$$
$$\tilde{a_2} = -\frac{1}{2} \, e_1 + \frac{1}{2} \sqrt{3} \, e_2$$
\item Man beweise, dass $\tilde{a_1}$ und $\tilde{a_2}$ linear unabhängig sind. Dass ist $\widetilde{B} = (\tilde{a_1}, \tilde{a_2})$ eine Basis von $\mathbb{R}^2$.
\item Man bestimme für die Drehung $f \in Hom(V, V)$ um $0_V$ im Gegenuhrzeigersinn um den Winkel $\varphi = 90^{\circ}$ die Matrix $M(f; \widetilde{B}, \widetilde{B})$. Man vergleiche diese Matrix mit der in \textbf{Aufgabe 31} bestimmten Matrix $M(f; B, B)$.
\end{enumerate}
\end{aufgabe}


\begin{aufgabe}{?????}
Eine Reisegesellschaft aus 16 Personen verbringt schöne Tage in einem Ferienhotel, in dessen Speisesaal 4 Tische mit jeweils 4 Stühlen stehen. Es wird der Wunsch geäußert, zu jeder Mahlzeit die Tischordnung so zu ändern, bis jeder mit jedem einmal an einem Tisch gesessen hat. Man gebe Tischordnungen an, bei deren vollständiger Umsetzung es tatsächlich gelingt, den Wunsch zu erfüllen. \vspace{3mm} \\
Hilfe: Man identifiziere die 16 Personen mit den 16 Elementen von $K^2$, wobei $K$ Körper mit vier Elementen wie in \textbf{Aufgabe 13 b)}. Man greife dann auf die \textbf{Aufgaben 23} und \textbf{10} zurück; für jeden eindimensionalen Untervektorraum $U$ von $K^2$ liefert die in \textbf{Aufgabe 10} eingeführte Äquivalenzrelation eine Partition von $K^2$, d.h. $K^2$ ist disjunkte Vereinigung der zugehörigen Äquivalenzklassen. So erhält man eine Tischordnung.
\end{aufgabe}



\end{document}