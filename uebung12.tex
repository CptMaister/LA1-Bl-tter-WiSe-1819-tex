\documentclass{uebblatt}
\usepackage{enumitem}

\begin{document}

\maketitle{12}

\begin{aufgabe}{ (2+3=5 Punkte)}
$K$ sei ein Körper. Zwei Matrizen $M, \widetilde{M} \in K^{n \times m}$ heißen genau dann äquivalent ($M \sim \widetilde{M}$), wenn es $R \in GL(n; K), S \in GL(m; K)$ mit $\widetilde{M} = RMS$ gibt.
\begin{enumerate}
\item Man zeige , dass „$\sim$“ eine Äquivalenzrelation ist.
\item Man beweise: Zwei Matrizen $M, \widetilde{M} \in K^{n \times m}$ sind genau dann die darstellenden Matrizen einer Abbildung $A \in Hom_K(V, W) \> (dimV = m, dimW = n)$ bezüglich gewisser Basen $B_V$, $\widetilde{B}_V$ in $V$ und $B_W$, $\widetilde{B}_W$ in $W$, wenn sie äquivalent sind. Das heißt:
$$M = M(A; B_V, B_W) \land \widetilde{M} = M(A; \widetilde{B}_V, \widetilde{B}_W) \Leftrightarrow M \sim \widetilde{M}$$
Hilfe: Man denke an die Transformationsformel. Insbesondere zum Beweis von "$\Leftarrow$":\\
Man wähle $B_V$ und $B_W$ und definiere $A \in Hom_K(V, W)$ durch $M$ so, dass\break 
$M = M(A; B_V, B_W)$ gilt. Wegen $\widetilde{M} = RMS$ kann man mit Hilfe von $R^{-1}$ und $S$ Basen $\widetilde{B}_V$ und $\widetilde{B}_W$ definieren, so dass $\widetilde{M} = M(A; \widetilde{B}_V, \widetilde{B}_W)$ gilt.
\end{enumerate}
\end{aufgabe}


\begin{aufgabe}{ (3 Punkte)}
$K$ sei ein Körper. Auf $K^{n \times m}$ sei die Äquivalenzrelation wie in \textbf{Aufgabe 36} definiert. Wie viele verschiedene Äquivalenzklassen gibt es in $K^{n \times m}$? Man beweise die Antwort.\vspace{2mm} \\
Hinweis: Man zeige für Matrizen $M$ und $M \in K^{n \times m}: M \sim M \Leftrightarrow Rang(M) = Rang(\widetilde{M})$. Dabei denke man daran, dass es zu jedem $f \in Hom_K(K^m, K^n)$ mit $Rang(f) = r$ Basen $B_m$ in $K^m$ und $B_n$ in $K^n$ gibt, so dass $M(f; B_m, B_n) = 
\begin{pmatrix} E_r && 0 \\ 0 && 0 \end{pmatrix}$ („Normalform von $f$“).

\end{aufgabe}


\begin{aufgabe}{ (2+1+1=4 Punkte)}
Es sei $K$ ein Körper und es seien $V, W$ $K$-Vektorräume mit $dimV = m$, $dimW = n$. Es sei $0 \neq A \in Hom_K(V,W)$; wegen $A \neq 0$ ist dann $Ker(A) \neq V$. $U \neq 0$ sei ein Untervektorraum von $V$ derart, dass $V = U \oplus Ker(A)$.\\
Es ist folglich $dimU = dimV – dimKer(A) = dimBi(A)$.\\
Wir betrachten Basen $(a_1, ..., a_r)$ von $U$, $(a_{r+1}, ..., a_m)$ von $Ker(A)$, wobei\break
$r \defeq dimU = dimBi(A)$.
\begin{enumerate}
\item Man zeige, dass $B_V \defeq (a_1, ..., a_r, a_{r+1}, ..., a_m)$ Basis von $V$ ist.\\
Hinweis: Für den besonderen Fall $r = m$ ist dies von vornherein klar.
\item Man zeige, dass $(Aa_1, ..., Aa_r)$ linear unabhängig in $W$ ist. ($Aa_k$ steht kurz für $A(a_k)$)
\end{enumerate}
Setze $b_1 \defeq Aa_1, ..., \; b_r \defeq Aa_r$ und ergänze $(b_1, ..., b_r)$ zu einer Basis $B_W = (b_1, ..., b_r, b_{r+1}, ..., b_n)$ von $W$.
\begin{enumerate}[resume]
\item Man zeige, dass $M(A; B_V, B_W) = \begin{pmatrix} E_r && 0 \\ 0 && 0 \end{pmatrix}$
\end{enumerate}
\underline{Ergänzende Bemerkung}:\\
Hier wird also ein Verfahren beschrieben, wie man Basen von $V$ und $W$ konstruieren kann, um $A$ in Normalform zu erhalten. Man mache sich bewusst, dass bei der Normalform die Informationen über die Abbildung $A$ in den Basen $B_V$ und $B_W$ stecken. 
\end{aufgabe}



\end{document}