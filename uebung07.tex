\documentclass{uebblatt}
\usepackage{enumitem}

\begin{document}

\maketitle{7}

\begin{aufgabe}{ (2+2+1=5 Punkte)}
Es sei $\mathbb{R}[X]$ der Polynomring über dem Körper $\mathbb{R}$ (mit einer Unbestimmten). $\mathbb{R}[X]$ ist ein $\mathbb{R}$-Vektorraum und $\mathbb{R}_n[X] \defeq \{f \in \mathbb{R}[X] \> | \> deg(f) \le n\}$ ist ein Untervektorraum davon. Man gebe eine Basis von
\begin{enumerate}
\item $\mathbb{R}[X]$
\item $\mathbb{R}_n[X]$
\end{enumerate}
an und beweise die Basiseigenschaft.\\ \\
\underline{Ergänzende Bemerkung}: \\
Wenn auch jeder $K$-Vektorraum eine Basis besitzt, so kann sich    die explizite Angabe einer Basis schwierig gestalten; ein typisches Beispiel dafür ist der $\mathbb{Q}$-Vektorraum $\mathbb{R}$. 
\begin{enumerate}[resume]
\item In den \textbf{Aufgaben 16} und \textbf{20} wurde der $\mathbb{R}$-Vektorraum $\mathbb{R}^\mathbb{N}$ thematisiert. Ist das in \textbf{Aufgabe 20} angegebene $B = (e_1, e_2, …)$ eine Basis des $\mathbb{R}$-Vektorraums $\mathbb{R}^\mathbb{N}$? Man begründe die Antwort.
\end{enumerate}
\end{aufgabe}


\begin{aufgabe}{ (2+2=4 Punkte)}
\begin{enumerate}
\item Man zeige: $k_2 = \{0, 1\}$ ist ein Unterkörper des Körpers $K = \{0, 1, a, b\}$ mit vier Elementen (siehe \textbf{Aufgabe 13 b)}) und K ist ein $k_2$-Vektorraum.
\item Man gebe eine Basis des $k_2$-Vektorraums $K$ an und beweise die Basiseigenschaft.
\end{enumerate}
\end{aufgabe}


\begin{aufgabe}{ (4 Punkte)}
Es ist $K$ der Körper mit 4 Elementen aus \textbf{Aufgabe 13 b)}. Dann ist $K^2$ ein $K$-Vektorraum mit 16 Vektoren. Unter Verwendung der Verknüpfungstafeln von $K$ gebe man alle eindimensionalen Untervektorräume von $K^2$ explizit an. \vspace{1.5mm} \\
\underline{Hinweis}: Ein Untervektorraum $U$ von $K^2$ ist genau dann eindimensional, wenn $U = K(\alpha_1, \alpha_2)$ für ein $(\alpha_1, \alpha_2) \neq (0, 0) \in K^2$. Im eben genannten Satz kann man natürlich $K(\alpha_1, \alpha_2)$ durch $span((\alpha_1, \alpha_2))$ ersetzen, denn $K(\alpha_1, \alpha_2) = span((\alpha_1, \alpha_2))$.\vspace{1.5mm} \\
Hilfe: Man überlege, wie die Antworten auf folgende Fragen aussehen: In wie vielen verschiedenen eindimensionalen Untervektorräumen von $K^2$ kann ein Element $(\alpha_1, \alpha_2) \neq (0, 0) \in K^2$ liegen? Wie viele verschiedene Elemente hat ein eindimensionaler Untervektorraum von $K^2$?
\end{aufgabe}



\end{document}