\documentclass{uebblatt}

\begin{document}

\maketitle{3}

\begin{aufgabe}{Zu den Eigenschaften von Äquivalenzklassen}
Es sei~$\sim$ eine Äquivalenzrelation auf einer Menge~$M$. Man beweise für Äquivalenzklassen~$[x]$ und~$[x']$ in~$M$:
\begin{enumerate}
\item ~$[x]$ und~$[x']$ sind disjunkt (d.h.~$[x] \cap [x'] = \emptyset$) oder gleich (d.h.~$[x] = [x']$).
\item ~$[x]$ und~$[x']$ sind genau dann gleich, wenn~$x \sim x'$ gilt.
\end{enumerate}
Hilfe zu a): Zu zeigen ist: Im Falle von ~$[x] \cap [x'] \neq \emptyset$ gilt ~$[x] = [x']$\\
\\
\underline{Ergänzende Bemerkung:}\\
Als Folge von a) und b) kommt jedes~$x \in M$ in genau einer Äquivalenzklasse vor.~$M$ ist somit die Vereinigungsmenge aller existierenden verschiedenen Äquivalenzklassen und wird dadurch in paarweise disjunkte Teilmengen zerlegt. Dies ist wertvoll, wie beispielsweise folgende Überlegung zeigt:\\
Bestünden alle existierenden Äquivalenzklassen aus gleich vielen verschiedenen Elementen - sagen wir~$n$ - und gäbe es~$k$ verschiedene Äquivalenklassen, dann bestünde~$M$ aus~$n \cdot k$ verschiedenen Elementen.\\
\end{aufgabe}

\begin{aufgabe}{Wichtiges Beispiel für eine Gruppe}
$M$ sei eine nichtleere Menge. Die Menge~$\gamma(M)$ aller bijektiven Abbildungen von~$M$ in sich selbst werde kurz mit~$G$ bezeichnet, d.h.~$G \defeq \{f:M \to M | f$ ist bijektiv$\}$.
\begin{enumerate}
\item Man gebe $G$ für $M = \{x_1, x_2\}$ und $M = \{x_1, x_2, x_3\}$ explizit an. Wie viele verschiedene Elemente hat $G$, wenn $M$ aus $n$ verschiedenen Elementen besteht?
\item Auf $G$ werde als Verknüpfung $\circ$ die Hintereinanderausführung der bijektiven Abbildungen definiert. Man erstelle die Verknüpfungstafeln von $G$ für\\$M = \{x_1, x_2\}$ und $M = \{x_1, x_2, x_3\}$.
\item Man verifiziere anhand der Verknüpfungstafeln, dass $G$ für diese Beispiele von $M$ mit dieser Verknüpfung $\circ$ eine Gruppe ist.
\item Man beweise: Für jede nichtleere Menge $M$ ist $G$ mit der Hintereinanderausführung als Verknüpfung eine Gruppe.
\end{enumerate}
\end{aufgabe}



\end{document}