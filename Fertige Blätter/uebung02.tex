\documentclass{uebblatt}

\begin{document}

\maketitle{2}

\begin{aufgabe}{Distributivität von Schnitt und Vereinigung}
$A$, $B$ und $C$ seien beliebige Mengen. Beweise:
\begin{enumerate}
\item ~$A \cap (B \cup C) = (A \cap B) \cup (A \cap C)$
\item ~$A \cup (B \cap C) = (A \cup B) \cap (A \cup C)$
\end{enumerate}
Tipp: Für beliebige Mengen X und Y gilt:~$X = Y \Leftrightarrow X \subset Y \wedge X \supset Y$\\
\end{aufgabe}

\begin{aufgabe}{Mengenoperationen unter Abbildungen}
$M$ und $N$ seien (nichtleere) Mengen,~$f:M \to N$ sei eine Abbildung.\\
Es gelte~$A_1 \subset M$, $A_2 \subset M$. Bei c) wird zusätzlich~$A_2 \subset A_1$ vorausgesetzt. Beweise:
\begin{enumerate}
\item ~$f(A_1 \cap A_2) \subset f(A_1) \cap f(A_2)$
\item ~$f(A_1 \cup A_2) = f(A_1) \cup f(A_2)$
\item ~$f(A_1 \setminus A_2) \supset f(A_1) \setminus f(A_2)$
\end{enumerate}
Interessant zu a) ist:\\
Es kann~$f(A_1 \cap A_2) \neq f(A_1) \cap f(A_2)$ sein. Ein Beispiel hierzu:\\~$M = N = \{0, 1, 2, 3\}, A_1 = \{0, 1\}, A_2 = \{2, 3\}$,\\
$ f:M \to N$ sei definiert durch~$0 \mapsto 0, 1 \mapsto 1, 2 \mapsto 1, 3 \mapsto 2$,\\
dann ist $A_1 \cap A_2 = \emptyset, f(A_1) \cap f(A_2) = \{0, 1\} \cap \{1, 2\} = \{1\}$.\\
\end{aufgabe}

\begin{aufgabe}{Mengenoperationen unter Abbildungen}
$M$ und $N$ seien (nichtleere) Mengen und~$f:M \to N$ sei eine Abbildung.\\
Es gelte~$B_1 \subset N$,~$B_2 \subset N$. Bei c) wird zusätzlich~$B_2 \subset B_1$ vorausgesetzt. Beweise:
\begin{enumerate}
\item ~$f^{-1}(B_1 \cap B_2) = f^{-1}(B_1) \cap f^{-1}(B_2)$
\item ~$f^{-1}(B_1 \cup B_2) = f^{-1}(B_1) \cup f^{-1}(B_2)$
\item~$f^{-1}(B_1 \setminus B_2) = f^{-1}(B_1) \setminus f^{-1}(B_2)$
\end{enumerate}
\end{aufgabe}

\end{document}